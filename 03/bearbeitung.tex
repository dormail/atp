\documentclass[11pt a4paper]{article}
\usepackage[margin=2cm]{geometry}
\usepackage{amsmath, amssymb}
\usepackage{graphicx}
\usepackage{float}
\usepackage{aligned-overset}
%\usepackage[normalem]{ulem} % fuer durchstreichen
\usepackage{cancel} % anderes durchstreichen

% partielle ableitungen
\newcommand{\delr}{\partial_r}
\newcommand{\deltheta}{\partial_\theta}
\newcommand{\delphi}{\partial_\varphi}

% elektrische feldkonstante
\newcommand{\epsz}{\epsilon_0}
% 1 / 4pi eps
\newcommand{\kco}{\frac{1}{4\pi\epsilon_0}}

% fancy header
\usepackage{fancyhdr}
\fancyhf{}
% vspaces in den headern fuer Distanzen notwendig
% linke Seite: Namen der Abgabegruppe

% zum zeichnen tikz
\usepackage{tikz}

\begin{document}
\thispagestyle{fancy}
\section*{Aufgabe 9}
\textbf{Neutrinos bei einer Kernkollapssupernova} \newline

Bei einer Kernkollapssupernova spielt der Elektroneneinfang eine große
Rolle. Dieser Prozess spielt eine wichtige Rolle bei der Bildung von 
Neutronensternen, wobei neben einem Neutron auch ein Elektron-Neutrino 
entsteht:
\[
	p + e^- \rightarrow n + v_e
\]

Die Prozesse beim CNO-Zyklus stellen natürlich auch eine Neutrinoquelle dar,
sind bei der kurzen Dauer einer Kernkollapssupernova aber eher unsignifikant.\\

Eine andere, wichtige Neutrinoquelle beim Kernkollaps stellen thermische 
Neutrinos dar. Für eine Stabilisierung des Neutronensterns muss dieser nach
der Supernova abkühlen, wobei die thermische Energie in Form von 
verschiedenen Neutrino-Antineutrino Paaren abgegeben wird.
\newline
\vspace{0.5cm}
\newline
\textbf{Neutrinos bei einer Supernova Ia}

Bei einer Supernova Ia finden verschiedene Fusionsprozesse statt, wobei 
einige davon auch Neutrinoquellen darstellen. Dazu gehören auch die 
Fusionsreaktionen bei der p-p-Kette und beim CNO-Zyklus.

Eine weitere Neutrinoquelle sind durch Paarbildung entstehende 
Neutrino-Antineutrino-Paare. Diese sind allerdings sehr unwahrscheinlich 
($P = 10^{-19}$).
\newline
\vspace{0.5cm}
\newline
\textbf{Vergleich: Neutrinos bei Kernkollaps- und Typ Ia Supernovae}

Da beim Kernkollaps ein sehr großer Teil des Sternes zu Neutronen wird, 
muss der dabei relevante Vorgang, der Elektroneneinfang, auch in 
entsprechender Häufigkeit stattfinden. Da dabei auch immer ein Neutrino 
entsteht, ist der Kernkollaps mit einer hohen Neutrinoemision verbunden.
\\
Die Neutrinoquellen, die bei einer Supernova Ia vorkommen sind nicht 
nur schwächere 
Neutrinoquellen; da bis kurz vorm Kernkollaps Fusionsprozesse stattfinden,
liegen bei einer Kernkollapssupernova auch bis kurz vorm Kollaps die damit 
verbundenen Neutrinosemissionen vor.
\\
Da die Prozesse mit Neutrinoemission beim Kernkollaps vermehrt auftreten,
ist dabei mit mehr Neutrinos zu rechnen als bei einer Typ Ia Supernova.

\end{document}
