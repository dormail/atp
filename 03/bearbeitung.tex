\documentclass[11pt a4paper]{article}
\usepackage[margin=2cm]{geometry}
\usepackage{amsmath, amssymb}
\usepackage{graphicx}
\usepackage{float}
\usepackage{aligned-overset}
%\usepackage[normalem]{ulem} % fuer durchstreichen
\usepackage{cancel} % anderes durchstreichen

% partielle ableitungen
\newcommand{\delr}{\partial_r}
\newcommand{\deltheta}{\partial_\theta}
\newcommand{\delphi}{\partial_\varphi}

% elektrische feldkonstante
\newcommand{\epsz}{\epsilon_0}
% 1 / 4pi eps
\newcommand{\kco}{\frac{1}{4\pi\epsilon_0}}

% fancy header
\usepackage{fancyhdr}
\fancyhf{}
% vspaces in den headern fuer Distanzen notwendig
% linke Seite: Namen der Abgabegruppe
\lhead{\textbf{Etem Kalyon\\Matthias Maile\\Roman Surma}\vspace{1.5cm}}
% rechte Seite: Modul, Gruppe, Semester
\rhead{\textbf{Physik II - Gruppe 2\\Sommersemester 2020}\vspace{1.5cm}}
% Center: nr. des blattes
\chead{\vspace{2.5cm}\huge{\textbf{16. Übungsblatt}}}
% benoetigt damit der eigentliche Text nicht in der Überschrift steckt
\setlength{\headheight}{4cm}

% zum zeichnen tikz
\usepackage{tikz}

\begin{document}
\thispagestyle{fancy}
\section*{Aufgabe 7}
\par{a)}
Aus dem Druck Gradienten können wir durch Integrieren den Druck bestimmen:
\begin{align*}
	\text{Druckgradient: }
	dP(r) &= -\frac{GM \rho(r)}{r^2} dr \hspace{2cm}
	M(r) = \frac43 \pi r^3 \rho \\
	\Leftrightarrow
	dP(r) &= -\frac{4 \pi r^{\cancel{3}} \rho^2 G}{\cancel{r^2}} dr \\
	% integral setzen
	\Rightarrow
	P(r) = \int dP(r)
	&= - \int_R^r G \frac43 \pi r^\prime \rho^2 dr^\prime \\
	% loesen
	\Rightarrow
	P(r) &= - G \frac23 \pi \rho^2 r^{\prime 2} \vert_R^r \\
	% einsetzen in stmmfkt
	P(r) &= \frac23 \pi G \rho^2 (R^2 - r^2)
\end{align*}


\section*{Aufgabe 8}
\section*{Aufgabe 9}

\end{document}
