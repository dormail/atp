\documentclass[11pt, a4paper]{article}
\usepackage[margin=2cm]{geometry}
\usepackage{amsmath, amssymb}
\usepackage{graphicx}
\usepackage{float}
\usepackage{aligned-overset}

% partielle ableitungen
\newcommand{\delr}{\partial_r}
\newcommand{\deltheta}{\partial_\theta}
\newcommand{\delphi}{\partial_\varphi}

% elektrische feldkonstante
\newcommand{\epsz}{\epsilon_0}
% 1 / 4pi eps
\newcommand{\kco}{\frac{1}{4\pi\epsilon_0}}

% fancy header
\usepackage{fancyhdr}
\fancyhf{}
% vspaces in den headern fuer Distanzen notwendig
% linke Seite: Namen der Abgabegruppe
\lhead{\textbf{Benedikt Sander \\Tahir Kamcili \\ Matthias Maile}\vspace{1.5cm}}
% rechte Seite: Modul, Gruppe, Semester
\rhead{\textbf{Astroteilchenphysik\\Sommersemester 2020}\vspace{1.5cm}}
% Center: nr. des blattes
\chead{\vspace{2.5cm}\huge{\textbf{6. Übungsblatt}}}
% benoetigt damit der eigentliche Text nicht in der Überschrift steckt
\setlength{\headheight}{4cm}

% zum zeichnen tikz
\usepackage{tikz}

\begin{document}
\thispagestyle{fancy}
\noindent
{\large\textbf{Aufgabe 17}} \\[0.2cm]

a) Der CMB ist die erste (bzw. älteste) beobachtbare Strahlung im Universum. \\
Aufgrund der hohen (Energie- und Materie-) Dichte kurz nach dem Urknall konnten sich Protonen und 
Elektronen nicht binden. Durch diese mit freien Elelektronen gefüllte ``Wolke`` konnten sich aufgrund 
der Thomson-Streuung Photonen nicht ausbreiten.\\
Erst als sich das Universum so weit ausgebreitet hatte, dass die Temperatur unter $3000K$ fiel, 
(ca. {380000} Jahre nach dem Urknall) konnten sich 
Wasserstoffatome bilden und elektromagnetische Strahlung konnte sich ausbreiten.\\
Da sich das Universum weiter ausgedehnt hat, fiel auch die dem CMB entsprechende Temperatur (wenn man annimmt 
die Strahlung käme von einem schwarzen Strahler) weiter runter, mittlerweile enstpricht dem CMB eine Temperatur von
$2.725K$. 
Diese Strahlung liegt im Mikrowellenbereich und besitzt ihre maximale Stärke bei $\lambda = 1.06\ \text{mm}$.
\\
Aus der Differenz zwischen aktueller Temperatur des CMB, der $3000K$ Grenze und der kosmologischen Rotverschiebung
rührt dann ein Skalenfaktor $z=1089$ her, welcher die Ausdehnung des Universums seit der Enkopplung von Materie und 
Strahlung beschreibt:
\[ 3000K \approx 2.725K \cdot (1 + z) \]

b) Das Multipolspektrum trägt die Temperaturfluktationen (bzw. die damit verbundenen \\ Fluktuationen des CMB) 
gegen die Bogenlänge, auf der diese Fluktuation stattfindet, auf. \\
Da hier über $m$ (Anzahl Moden entlang es Äquators) der durchschnitt gebildet wurde kann man die mittlere 
Wellen-``länge`` (die hier einem Winkel entspricht) abschätzen mit
\[ \lambda \approx \frac{\pi}{l}. \]
Wobei auch der Ersteller des Diagramms diese Abschätzung (obere und untere $x$-Achsen Skala) benutzt hat. \\
Der Peak bei $\sim 1^\circ$ bzw. $l = 200$ bedeutet zum Beispiel, dass wenn man einen Bereich am Himmel
mit einem Radioteleskop mit einer  Begrenzung auf $1^\circ$ ``Sichtfeld``
betrachtet, starke Fluktuationen (im CMB Spektrum) sieht.

\newpage
c) Silk-Dämpfung:\\
Die Silk-Dämpfung sorgte im frühen Universum (vor der Entkopplung) für eine bessere homogenisierung des 
Universums. \\
Dabei diffundierten Photonen von warmen  (und damit massereicheren) Regionen des Universums in Kältere. 
\begin{center}
\begin{tikzpicture}
	% ovale und anotation
	\draw[draw=red] (-3,0) circle [x radius = 2, y radius = 1]
		(-3,1.2) node[red] {Warme Region};
	\draw[draw=blue] (3,0) circle [x radius = 2, y radius = 1]
		(3,1.2) node[blue] {Kalte Region};
	% label photonen
	\draw (0, 0.85) node {Diffundierende};
	\draw (0, 0.5) node {Photonen};
	% pfeile mit label 
	\draw [->] (-1.5, 0.3) -- (1.5, 0.3);
	\draw [->] (-1.5, 0.0) node[left] {$\gamma$} (-1.5, 0.0) -- (1.5, 0.0);
	\draw [->] (-1.5, -0.3) -- (1.5, -0.3);
\end{tikzpicture}
\end{center}
Die Photonen ziehen dabei Elektronen mit sich, welche mit der Coulombkraft auch Protonen mit sich ziehen.\\
Damit findet insgesamt ein Materiefluss von massereichen zu massearmen Regionen statt, welcher gegen Anisotropien 
``vorgeht``.
\newline
\vspace{0.0cm}
\\
Nicht-integrierter Sachs-Wolfe-Effekt \\
Dadurch dass zum Zeitpunkt der Entkopplung keine 100\%-ige Isotropie im Universum vorlag, gab es Stellen 
gebiete mit höherer und niedrigerer Massedichte und damit
höheren und niedrigeren Gravitationspotentialen. Diese Potentialunterschiede sorgen bei den Photonen,
welche aus den entsprechenden Regionen stammen, für eine Rot- bzw. Blauverschiebung, welche aus der 
allgemeinen Relativitätstheorie folgt.


\end{document}
