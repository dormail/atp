\documentclass[a4paper 11pt]{article}
\usepackage{amsmath, amssymb}
\usepackage[margin=2.5cm]{geometry}

\begin{document}
\textbf{\large Aufgabe 2:} 
\paragraph{a)} 
Das Stefan-Boltzman-Gesetz:
\[
	P = \sigma * A * T^4
\]
Für die Sonne ergibt sich:
\begin{align*}
	P_\odot = \sigma * A_\odot * T_\odot^4
	&= \sigma * 4\pi R_\odot^2 * T_\odot^4 \\
	&= 5.67  * 10^{-8} * 4\pi * 6.96^2 * 10^{16} * 5.78^4 * 10^{12} W
	\\
	&= 3.8523 * 10^{26} W
\end{align*}
Für den Fluss in einer Distanz von 1 au ergibt sich:
\[
	P_{\odot, \text{au}}
	= P_\odot \frac{1}{4\pi(\text{au})^2}
	= 1.3698 * 10 ^3 \frac{W}{m^2}
	\]

\paragraph{b)}
Die Querschnittsfläche der Erde (da die Erde ja immer nur von einer Seite 
bestrahlt wird):
\[
	A_{\text{Erde}, \vert} = \pi * R_{\text{Erde}}^2 
	= \pi * 6.36^2 * 10^6 m^2
	= 1.2708 * 10^8 m^2
\]
Mit der Albedo lässt sich die absorbierte Leistung ermitteln:
\[
	P_{\text{absorb.}} =
	P_{\odot, \text{au}} * A_{\text{Erde}, \vert} * (1 - \text{Albedo})
	=
	1.2185 * 10^{11} W
\]
\paragraph{c)}
Das Stefan-Bolzman-Gesetz lässt sich zur Temperatur umstellen:
\[
	P = \sigma * A * T^4 \Rightarrow
	T = \sqrt[4]{\frac P {\sigma A}}
\]
Dabei soll die Strahlungsleistung der Erde (oben mit P bezeichnet) der absorbierten Strahlung entsprechen. Dann ergibt sich:
\begin{align*}
	T 
	&= \sqrt[4]{\frac{ P_{\odot, \text{au}} * A_{\text{Erde}, \vert} * 
	(1 - \text{Albedo})}{\sigma A_{\text{Erde}, \circ}}} \\
	% flaechen einsetzen
	&= \sqrt[4]{\frac{ P_{\odot, \text{au}} * \pi * R_{\text{Erde}}^2 * (1-\text{Albedo}) }
	{\sigma * 4\pi R_{\text{Erde}}^2 }} \\
	% kuerzen und weiter einsetzen
	&= \sqrt[4]{
		P_\odot \frac{1}{4\pi(\text{au})^2}
		\frac{(1-\text{Albedo})}{4 \sigma}} \\
	% P_sonne einsetzen
	&= \sqrt[4]{
		\sigma * 4\pi R_\odot^2 * T_\odot^4 
		\frac{1}{4\pi(\text{au})^2}
		\frac{(1-\text{Albedo})}{4 \sigma}} \\
	% kuerzen
	&= T_\odot \sqrt[4]{
		\left( \frac{R_\odot}{\text{au}} \right)^2
		\frac{(1-\text{Albedo})}{4}}
	% ergebenis
	\approx 255 K
\end{align*}

\paragraph{d)}
Durch die Atmosphäre und den damit verbundenen Treibhauseffekt wird ein 
Teil der Wärmestrahlung der Erde direkt wieder zurückgeworfen. Die Erde
muss also wärmer sein, um die selbe Menge an Wärme aus der Atmosphäre ``raus
zu befördern.``\\
Dadurch liegt die tatsächliche Durschnittstemperatur über der berechneten 
Gleichgewichtstemperatur.

\end{document}
