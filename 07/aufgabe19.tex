
\documentclass[11pt, a4paper]{article}
\usepackage[margin=2cm]{geometry}
\usepackage{amsmath, amssymb}
\usepackage{graphicx}
\usepackage{float}
\usepackage{aligned-overset}

% partielle ableitungen
\newcommand{\delr}{\partial_r}
\newcommand{\deltheta}{\partial_\theta}
\newcommand{\delphi}{\partial_\varphi}

% elektrische feldkonstante
\newcommand{\epsz}{\epsilon_0}
% 1 / 4pi eps
\newcommand{\kco}{\frac{1}{4\pi\epsilon_0}}

% fancy header
\usepackage{fancyhdr}
\fancyhf{}
% vspaces in den headern fuer Distanzen notwendig
% linke Seite: Namen der Abgabegruppe
\lhead{\textbf{Benedikt Sander \\Tahir Kamcili \\ Matthias Maile}\vspace{1.5cm}}
% rechte Seite: Modul, Gruppe, Semester
\rhead{\textbf{Astroteilchenphysik\\Sommersemester 2020}\vspace{1.5cm}}
% Center: nr. des blattes
\chead{\vspace{2.5cm}\huge{\textbf{7. Übungsblatt}}}
% benoetigt damit der eigentliche Text nicht in der Überschrift steckt
\setlength{\headheight}{4cm}

% zum zeichnen tikz
\usepackage{tikz}

\begin{document}
\thispagestyle{fancy}
\noindent
{\large\textbf{Aufgabe 19}} \\[0.2cm]
a) Da sich in beiden Prozessen jeweils ein Teilchen in zwei Aufteilt, verdoppelt sich nach jedem Splitting
die Teilchenanzahl:
\[ N_n = 2^n \]
Die Energie wird hälftig aufgeteilt, d.h. dass die Energie pro Teilchen nach $n$ Splits gegeben ist durch
\[ E_n = \frac{E_0}{N_n} = E_0 \cdot 2^{-n}. \]
b) Der Teilchenschauer geht so lange weiter, bis $E_n$ unter die kritische Energie $E_c$ fällt. Die Anzahl an
Splitting Events ist somit vom logarithmus der Anfangsenergie abhängig:
\[ 
	E_n = E_0 \cdot 2^{-n} \overset !> E_c 
	\Rightarrow
	n_\text{max} = \text{log}_2 \left( \frac{E_0}{E_c} \right)
\]
Daraus folgt die maximale Anzahl an Teilchen:
\[
	N_\text{max} = 2^{n_\text{max}} = 2^{\text{log}_2 (E_0 / E_c )} = \frac{E_0}{E_c} 
\]
Wenn man die Splitting-Länge $d$ kennt lässt sich die atmosphärische Tiefe errechnen:
\[ X_\text{max}^\gamma = d \cdot n_\text{max} = d \cdot \text{log}_2 \left( \frac{E_0}{E_c} \right) \]
c) 
\[ n_\text{max} = \text{log}_2 \left( \frac{E_0}{E_c} \right) = 13.522 \]
\[ X_\text{max}^\gamma = d \cdot n_\text{max} = 338.05 \frac{\text{g}}{\text{cm}^2} \]
d)
\begin{align*}
	X
	&= \int \rho(h) \ dh \\
	% einsetzen
	&= \int_{h_\text{max}}^\infty \rho(0) \exp \left(-\frac{h}{h_s} \right) \ dh \\
	% integrieren
	&= \left. -h_s \ \rho(0) \exp \left(-\frac{h}{h_s} \right) \right|_{h_\text{max}}^\infty \\
	% einsetzen
	&= h_s \ \rho(0) \exp \left(-\frac{h_\text{max}}{h_s} \right) \\
	\Rightarrow
	\exp \left(-\frac{h_\text{max}}{h_s} \right)
	&= \frac{X}{h_s \ \rho(0)} \\
	% ln setzen
	\Rightarrow
	-\frac{h_\text{max}}{h_s}
	&= \ln \left( \frac{X}{h_s \ \rho(0)} \right) \\
	% endergebnis
	\Rightarrow
	h_\text{max}
	&= -h_s \ \ln \left( \frac{X}{h_s \ \rho(0)} \right) \\
	&\approx 9.57 \ \text{km}
\end{align*}

\newpage
\setlength{\headheight}{0cm}

e) Ein hadronischer Schauer wird, wie der Name schon vermuten lässt durch Hadronen ausgelöst. \\
Beim hadronischen Schauer entstehen dabei sehr viele Pionen, wobei neutrale Pionen in 2 Photonen zerfallen
und dadurch einen elektromagnetischen Schauer auslösen:
\[ \pi^0 \rightarrow \gamma + \gamma \]
f) Ein hadronischer Schauer besitzt ein Profil welches sich deutlich von dem eines Elektromagnetischen 
unterscheidet. \\
Da bei hadronischen Schauern die Teilchen in unterschiedliche Richtungen abgelenkt werden können, besitzen
diese ein ``chaotischeres`` Muster, wo es zwar einen Kernbereich gibt, aber auch außerhalb davon treffen
Teilchen auf die Detektoren.\\
Die Signatur eines elektromagnetischen Schauers ist ``fokussierter``, es gibt beim Detektor einen ``Lichtkegel``,
in dem viel Cherenkov-Strahlung messbar ist, außerhalb von diesem jedoch sehr wenig passiert.

\end{document}
