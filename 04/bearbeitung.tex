\documentclass[11pt, a4paper]{article}
\usepackage[margin=2cm]{geometry}
\usepackage{amsmath, amssymb}
\usepackage{graphicx}
\usepackage{float}
\usepackage{aligned-overset}

% partielle ableitungen
\newcommand{\delr}{\partial_r}
\newcommand{\deltheta}{\partial_\theta}
\newcommand{\delphi}{\partial_\varphi}

% elektrische feldkonstante
\newcommand{\epsz}{\epsilon_0}
% 1 / 4pi eps
\newcommand{\kco}{\frac{1}{4\pi\epsilon_0}}

% fancy header
\usepackage{fancyhdr}
\fancyhf{}
% vspaces in den headern fuer Distanzen notwendig
% linke Seite: Namen der Abgabegruppe
\lhead{\textbf{Benedikt Sander \\Tahir Kamcili \\ Matthias Maile}\vspace{1.5cm}}
% rechte Seite: Modul, Gruppe, Semester
\rhead{\textbf{Astroteilchenphysik\\Sommersemester 2020}\vspace{1.5cm}}
% Center: nr. des blattes
\chead{\vspace{2.5cm}\huge{\textbf{4. Übungsblatt}}}
% benoetigt damit der eigentliche Text nicht in der Überschrift steckt
\setlength{\headheight}{4cm}

% zum zeichnen tikz
\usepackage{tikz}

\begin{document}
\thispagestyle{fancy}
\noindent
{\large\textbf{Aufgabe 12}} \\[0.2cm]
a) Für die Bestimmung der Hubble-Konstante $H_0$ gibt es zwei Methoden, 
welche allerdings zwei abweichende Ergebnisse liefert.\\
Eine Methode ist es, mit der kosmischen Hintergrundstrahlung die Hubble
Konstante zu ermittlen. Dabei erhält man einen Wert für $H_0$:
\[
	H_0 = 66.6 \ km \ s^{-1} Mpc^{-1}
\]
Eine andere Methode ist es, von einem entfernten Objekt mit der 
kosmischen Abstandsleiter die Distanz zu bestimmen
und dann ausgehend von der 
Distanz und der Rotverschiebung die Hubble-Konstante zu errechnen.
\\
Damit erhält man einen (vom vorherigen Wert stark abweichenden) Wert:
\[
	H_0 = 74.03  \ km \ s^{-1} Mpc^{-1}
\]
Die Differenz zwischen den Werten ist sehr signifikant, und bei der 
Präzision der Messungen auch nicht mit Messfehlern erklärbar. \\
Da der Hubble-Parameter nicht konstant ist, könnte man eine 
solche Differenz mit dieser zeitlichen Abhängigkeit erklären. Allerdings 
weichen die Messwerte auch bei gleichen Messzeiträumen ab. Daher muss 
es noch weitere Effekte geben, welche noch nicht im kosmologischen 
Standartmodel einbezogen werden. 
\\[0.2cm]
Aus dem Zusammenhang $v = H_0 d$ lässt sich das Alter des Universums $t_0$
abschätzen:
\begin{align*}
	v &= H_0 d \\
	% integral setzen
	\int_0^{t_0} v \ dt &= \int_0^{t_0} H_0 d \ dt \\
	% integral berechnen
	d &= t_0 \cdot H_0 d \\
	% umstellen
	\Rightarrow 
	t_0 &= \frac{1}{H_0}
\end{align*}
Mit dieser Abschäzung erhalten wir für die eben genannten Werte:
\begin{align*}
	&\text{Hintergrund-Strahlung } &
	t_0 &= \frac{1}{66.6} \cdot 3.09 \cdot 10^{19} s
	= 14.8 * 10^9 \text{ Jahre}
	\\
	&\text{Distanzleiter } &
	t_0 &= \frac{1}{74.03} \cdot 3.09 \cdot 10^{19} s
	= 13.4 * 10^9 \text{ Jahre}
\end{align*}
\vspace{0.5cm}
b) Aus der Kosmischen Rotverschiebung folgen $a(t)$ und $\dot a(t)$:
\begin{align*}
	a(t) \cdot (1+z) = a_0 \Rightarrow 
	a(t) &= \frac{a_0}{1+z} \\
	% ableitung mit quotientenregel
	d_t a(t) &= a_0 \frac{d_tz}{(1+z)^2}
\end{align*}
\newpage
\setlength{\headheight}{0cm}
Setzen wir das in $H$ ein erhalten wir:
\begin{align*}
	H(a(t)) = \frac{d_t a}{a} 
	&= \frac{1+z}{a+0} \frac{d_tz \cdot a_0}{(1+z)^2} \\
	% kuerzen
	&= \frac{d_tz}{1+z} \tag{1}
\end{align*}
Aus der ersten Friedmann-Gleichung folgt zudem der Zusammenhang zwischen
dem Hubble-Parameter $H$ und der kosmischen Rotverschiebung $z$:
\[
	H(z) = H_0 \sqrt{
		\Omega_\text{Masse} (1+z)^3 + 
		\Omega_\text{Strahlung} (1+z)^4 +
		\Omega_\text{Dunkle Energie}}
\]
Wenn wir diesen Ausdruck mit dem aus $(1)$ gleichsetzen erhalten wir den
differentiellen Zusammenhang zwischen der Zeit und der kosmischen 
Rotverschiebung:
\begin{align*}
	\frac{d_tz}{1+z} \overset{!}&{=}
	H(z) = H_0 \sqrt{
		\Omega_\text{Masse} (1+z)^3 + 
		\Omega_\text{Strahlung} (1+z)^4 +
		\Omega_\text{Dunkle Energie}} \\
	% umstellen, differential ausschreiben
	\frac{dz}{dt} &=
	H_0 \sqrt{
		\Omega_\text{Masse} (1+z)^5 + 
		\Omega_\text{Strahlung} (1+z)^6 +
		\Omega_\text{Dunkle Energie} (1+z)^2} \\
	% differential umstellen
	dt &=
	\frac{dz}{ H_0 \sqrt{
		\Omega_\text{Masse} (1+z)^5 + 
		\Omega_\text{Strahlung} (1+z)^6 +
		\Omega_\text{Dunkle Energie} (1+z)^2}}
\end{align*}
\vspace{0.5cm}
c) Die Vereinfachung, das Universum bestehe nur aus Materie liefert uns:
\[
	\Omega_\text{Masse} = 1 \quad
	\Omega_\text{Strahlung} = 0 \quad
	\Omega_\text{Dunkle Energie} = 0 
\]
Damit können wir das Alter des Universums berechnen:
\begin{align*}
	t_0 &=
	\int_0^{t_0} dt =
	\int_{z(t = 0)}^{z = z(t = t_0)}
	\frac{dz}{ H_0 \sqrt{
		\Omega_\text{Masse} (1+z)^5 + 
		\Omega_\text{Strahlung} (1+z)^6 +
		\Omega_\text{Dunkle Energie} (1+z)^2}} \\
	% Omegae einsetzen
	&= \int_{z(t = 0)}^{z = z(t = t_0)}
	\frac{dz}{ H_0 \sqrt{(1 + z)^5 } } \\
	% stammfkt bilden
	&= \frac{1}{H_0} \left[ \frac{-1}{1.5} (1 + z)^{-1.5} \right]
	_{z(t=0)}^{z=z(t=t_0)} \\
	% definition von z einsetzen
	&= \frac{1}{H_0} 
	\left[ \frac{-1}{1.5} \left(\frac{R(t)}{R(t_0)}\right)^{1.5}
	\right]_{t=0} ^{t=t_0} \\
	% einsetze
	&= \frac{1}{H_0} \left( \frac{-1}{1.5} 
	\underbrace{\left(\frac{R(0)}{R(t_0)}\right)^{1.5}}_{=0}
	- \frac{-1}{1.5} 
	\underbrace{\left(\frac{R(t_0)}{R(t_0)}\right)^{1.5}}_{=1} 
	\right) \\
	% ergebnis
	&= \frac{2}{3  H_0}
\end{align*}


\end{document}
