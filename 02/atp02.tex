\documentclass[11pt, a4paper]{article}
\usepackage[margin=2cm]{geometry}
\usepackage{amsmath, amssymb}
\usepackage{graphicx}
\usepackage{float}
\usepackage{aligned-overset}

% partielle ableitungen
\newcommand{\delr}{\partial_r}
\newcommand{\deltheta}{\partial_\theta}
\newcommand{\delphi}{\partial_\varphi}

% elektrische feldkonstante
\newcommand{\epsz}{\epsilon_0}
% 1 / 4pi eps
\newcommand{\kco}{\frac{1}{4\pi\epsilon_0}}

% fancy header
\usepackage{fancyhdr}
\fancyhf{}
% vspaces in den headern fuer Distanzen notwendig
% linke Seite: Namen der Abgabegruppe
\lhead{\textbf{Benedikt Sander \\Tahir Kamcili \\ Matthias Maile}\vspace{1.5cm}}
% rechte Seite: Modul, Gruppe, Semester
\rhead{\textbf{Astroteilchenphysik\\Sommersemester 2020}\vspace{1.5cm}}
% Center: nr. des blattes
\chead{\vspace{2.5cm}\huge{\textbf{2. Übungsblatt}}}
% benoetigt damit der eigentliche Text nicht in der Überschrift steckt
\setlength{\headheight}{4cm}

% zum zeichnen tikz
\usepackage{tikz}

\begin{document}
\thispagestyle{fancy}
{\large\textbf{Aufgabe 4}}
\par{a)}
Die Masse eines Hohlkugel-Ausschnits aus der Sonne kann angegeben werden als
\[
	dM = \rho(r) 4\pi r^2 dr
\]
Aus dem Newtonschen Gravitationsgesetz folgt dann die Gewichtskraft, die
auf diese Kugelschale wirkt:
\[
	F_G = -G \frac{M_{\text{Sonne}} * dM}{r^2} 
	= -G \frac{M_{\text{Sonne}} * \rho(r) 4\pi r^2 dr}{r^2}
	= -GM 4\pi \rho(r) \ dr
\]
Die Kraft die eine Druckdifferenz $dP$ auf eine Fläche A bewirkt lässt sich 
schreiben als
\[
	F_p = A * dP = 4\pi r^2 dP
\]
Für ein hydrostatisches Gleichgewicht müssen diese Kräfte gleich sein.
Andernfalls würde sich der Stern ausdehnen bzw. zusammen ziehen, bis ein 
Gleichgeicht herscht.\\
Dadurch lässt sich der Druckgradient bestimmen:
\begin{align*}
	-GM 4\pi \rho(r) \ dr
	&= 4\pi r^2 dP \\
	% umformen
	\Rightarrow
	\frac{dP}{dr} &= -\frac{GM \rho(r)}{r^2}
\end{align*}

\vspace{0.5cm}
\par{b)}
Der Strahlungsfluss durch eine Kugelfläche entspricht der Leuchtkraft 
der Kugel geteilt durch die Kugelfläche:
\[
	F_{\text{rad}} = \frac{P}{A} = \frac{P}{4\pi r^2}
\]
\begin{center}
	{\footnotesize{(P entspricht hier jetzt  der Strahlungsleistung, 
	nicht dem Druck)}}
\end{center}
Wenn wir den Graviationsdruckgradienten mit dem dem Strahlungsgradienten 
gleichsetzen
erhalten wir die maximale Leuchtkraft (Eddington Leuchtkraft):
\begin{align*}
	-G \frac{M\rho(r)}{r^2} 
	&= -\frac{\kappa\rho(r)}{c} F_{\text{rad}} \\
	% F_rad einsetzen
	\Leftrightarrow
	-G \frac{M\rho(r)}{r^2} 
	&= -\frac{\kappa\rho(r)}{c} \frac{P}{4\pi r^2} \\
	% kuerzen
	\Leftrightarrow
	GM
	&= \frac{\kappa}{c} \frac{P}{4\pi} \\
	% nach p umstellen
	\Leftrightarrow
	P
	&= \frac{4\pi c M G}{\kappa}
\end{align*}

\vspace{0.5cm}
\par{c)}
Wenn die Leuchtkraft eines Sternes das Eddington Limit übersteigt, steigt 
der Strahlungsfluss $F_\text{rad}$, wodurch auch der Strahlungsdruckgradient
steigt. Über der Edington Grenze ist der Strahlungsdruck größer als der
Graviationsdruck, d.h. der Stern wird instabil und würde äußere Schichten 
ins Weltall ``abwerfen``.

\vspace{0.5cm}
\par{d)}
\[
	P_\text{Edd} = \frac{4\pi c MG}{\kappa} 
	= \frac{4\pi c  * 0.0083 * M_\odot G}{0.02m^2}kg
	\approx 2.1 * 10^{31} W
\]



\end{document}
